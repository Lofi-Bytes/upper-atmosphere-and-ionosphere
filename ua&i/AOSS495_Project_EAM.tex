\documentclass[12pt, letterpaper]{article}
\begin{document}

% Force pdflatex to use correct paper size.
\special{papersize=8.5in,11in}
\setlength{\pdfpageheight}{\paperheight}
\setlength{\pdfpagewidth}{\paperwidth}

% Set up title page.
\title{The Influence of Loss Terms on the Electron, Neutral, and Ion Temperatures}
\author{Erica A. Morgan\\AOSS 605 Final Project}
\date{April 26, 2011}

% Create both title page and contents.
% Use page breaks.
\maketitle
\newpage
\tableofcontents
\newpage

\begin{Abstract}
In order to improve upon the existing ionosphere structure developed in HW4, we will attempt to calculate the losses due to electron, ion, and neutral cooling.  We calculate these terms to develop more accurate equations for energy in a previously derived tridiagonal matrix designed to calculate temperatures for different particles.  By incorporating these equations we hope the code yields more realistic temperatures for the electron, ion, and neutral populations.
\end{abstract}

\section{Introduction}
When evaluating the dominant cooling processes for the thermal electron population in the ionosphere we must consider the differences in those processes as the altitude changes.  At higher latitudes the ionospheric plasma is nearly ionized.  As a result, the electrons undergo Coulomb collisions with the ions.  These collisions are the dominant energy loss mechanism in this region.  At lower latitudes the dominant energy loss mechanisms are rotational and vibrational excitation of the neutrals along with elastic collisions.  Another valid mechanism is the fine structure excitation of atomic oxygen.  Atomic oxygen could be excited to its lowest electronic state when the electron temperature is very high and could provide a significant loss as well.  However, this loss is not being considered for this particular evaluation.\\

\section{Background}
\subsection{Further Developing the Tridiagonal Matrix}
We begin by identifying the equation for calculating the appropriate loss terms.  We will employ equation 4.129c, one of the linear collision terms for the 13 moment approximation.

\begin{equation}
\frac{\delta E_{s}}{ \delta t} = - \Sigma \frac{n_{s}m_{s}\upsilon _{st}}{m_{s} + m_{t}} 3k(T_{s}-T_{t})
\end{equation}

Using known for the mass of an electron and the ion species of choice, we can achieve this.

\subsection{Calculating Collision Frequencies}
Equation 4.129c is dependent upon collision frequency.  We identify the appropriate collision frequency based on the nature of the interaction (ie. electron-ion, electron-neutral, etc.).  We consider necessary interactions and evaluate the known equations for each species taken from Shunk & Nagy.

For the electron(cool)-neutral(heat) interaction\upsilon _{en}

\begin{equation}
\upsilon _{eN_{2}} = 2.33\times 10^{-11}n(N_{2})(1-1.21\times 10^{-4}T_{e})T_{e}
\end{equation}


\begin{equation}
\upsilon _{eO_{2}} = 1.82\times 10^{-10}n(O_{2})(1+3.6\times 10^{-2}T_{e}^{1/2})T_{e}^{1/2}
\end{equation}


\begin{equation}
\upsilon _{eO} = 1.82\times 10^{-10}n(O)(1+3.57\times 10_{-4}T_{e})T_{e}^{1/2}
\end{equation}

In order to calculate collision frequencies due to the neutral-ion interactions [ion(cool)-neutral(heat)] the following equations apply\upsilon _{in}

\begin{equation}
\upsilon _{O^{+}O} = 3.67\times 10^{-11}n(O)T_{r}^{1\2}(1-0.064log_{10}T_{r})^{2}T_{e}^{1/2}
\end{equation}


\begin{equation}
\upsilon _{N_{2}^{+}N_{2}} = 5.14\times 10^{-11}n(N_{2})T_{r}^{1\2}(1-0.069log_{10}T_{r})^{2}T_{e}^{1/2}
\end{equation}


\begin{equation}
\upsilon _{O_{2}^{+}O_{2}} = 2.59 \times 10^{-11}n(O_{2})T_{r}^{1\2}(1-0.073log_{10}T_{r})^{2}T_{e}^{1/2}
\end{equation}

In these equations the average temperature is
\\
where T_{r} $is $

\begin{equation}
T_{r}= \frac{T_{i} + T_{n}}{2}
\end{equation}
\end{equation}

For the electron-ion interaction the following equation applies

\begin{equation}
\upsilon _{ei}= 54.5\frac{n_{i}Z_{i}^{2}}{T_{e}^{3/2}}
\end{equation}

Z, particle charge number, is equal to one in this instance but all ions are singly charged.

When calculating the ion-neutral interactions we must consider both resonant and non-resonant collision frequencies.  They are taken from Schunk & Nagy Table 4.5 and equation 4.146 respectively.

\begin{equation}
\upsilon _{in}= C_{in}n_{n}
\end{equation}

The coefficient is taken from Schunk & Nagy Table 4.4.

\subsection{Vibrational & Rotation Terms Chapter 9}
As we continue to paint a global picture of the overall heating and cooing mechanisms in the atmosphere we will include vibrational and rotation terms from chapter 9 in Schunk &Nagy.  We calculate these values for the N_{2},  O_{2}, and O_{Fine Structure} contributions.  As mentioned before, the O_{1D} excitation can be a notable species but will not be included into the overall calculation of the particle temperatures.  The equations are included below.


N_{2} $ rotation $ 

\begin{equation}
L_{e}(N_{2})=\frac{3.5\times 10^{-14}n_{e}n(N_{2})(T_{e}-T_{n})}{T_{e}^{1/2}} 
\end{equation}

O_{2} $rotation $

\begin {equation}
L_{e}(O_{2})=\frac{5.2\times 10^{-15}n_{e}n(O_{2})(T_{e}-T_{n})}{T_{e}^{1/2}} 
\end {equation}

O_{2} $vibration$

\begin{equation}
L_{e}(O_{2})=n_{e}n(O_{2})Q(T_{e})\{1-exp{[2239(T_{e}^{-1}-T_{n}^{-1})]\}
\end{equation}
\\
Oxygen fine structure 
\begin{equation}
\begin{eqnarray}
L_{e}(O)=n_{e}n(O)D^{-1}(S_{10}\{1-exp{[98.9(T_{e}^{-1}-T_{n}^{-1})]\} \\ && + S_{20} \{1-exp[326.6(T_{e}^{-1}-T_{n}^{-1})]\}\\
& &+ S_{21} \{1-exp[227.7(T_{e}^{-1}-T_{n}^{-1})]\})
\end{equation}
\end{eqnarray}

\section{Background}
Upon calculating all quantities we add or subtract these terms from our already developed equations for energy, Q, in the tridiagonal matrix.  Subtraction or addition of a given term is determined by the source of the exchange (ie. loss or gain)


\section{Results}
This is the plot of temperature versus time for the electron, ion, and neutral temperatures for half a day.

Insert Temperature Profile


\section{Discussion & Conclusion}







\end{document}